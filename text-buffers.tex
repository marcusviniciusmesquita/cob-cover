\startbuffer[book-title]
CULTIVANDO O BEM
\stopbuffer

\startbuffer[book-subtitle]
uma perspectiva budista para o cotidiano
\stopbuffer

\startbuffer[book-series]
budismo humanista
\stopbuffer

\startbuffer[honorary-title]
venerável mestre
\stopbuffer

\startbuffer[author]
Hsing Yün
\stopbuffer

\startbuffer[back-flap]
Hsing Yün, o Venerável Mestre, é o fundador da Ordem Budista Fo Guang Shan.
Quando ainda jovem, fez voto de revi\-talizar o Budismo Humanista.
Durante quase meio século, a força de seu voto possibilitou-lhe introduzir uma nova era para o Budismo Humanista.
Sua visão e seus esforços incansáveis influenciaram os estudos e práticas budistas em Taiwan e em todos os cinco continentes.

Reconhecido por seus métodos modernos, sistemáticos, ousados e inovadores de propagar os ensinamentos budistas para enfrentar as exigências contemporâneas, o Venerável Mestre Hsing Yün estabeleceu mais de 300 ramificações ao redor do mundo.
Fundou, também, quatro universidades públicas, dezesseis colégios budistas, vinte bibliotecas, duas gráficas, nove galerias de arte, uma clínica médica móvel e uma estação de televisão como meio de aproximar as pessoas do Budismo.

Em 1990, em Taiwan, República da China, fundou a Associação Internacional da Luz de Buda ({\sc blia}).
Em 1992, inaugurou a Sede Mundial da {\sc blia} nos {\sc eu\kern-1pta.}
Desde então, cerca de 100 capítulos internacionais foram instalados.
Ele é o principal arquiteto para construir a Terra Pura neste mundo.
\stopbuffer


\startbuffer[backcover-text]
O objetivo deste livro é simples: convidar o leitor a considerar o que significa levar uma boa vida e oferecer conselhos práticos, baseados nos ensinamentos budistas, sobre como alcançá-la.

Em cada um dos trinta e três ensaios, o Venerável Mestre Hsing Yün trata um assunto específico sobre moral ou ética, usando citações do rico tesouro das escrituras budistas como ponto de partida para sua argumentação.

Entre os tópicos que ele considera estão o controle do corpo e da fala, como superar a cobiça, erradicar a ira, ter paciência quando insultado, gerir a riqueza, conviver bem com os outros, o que significa praticar o Budismo e as bênçãos e alegrias dessa prática.
Os preceitos budistas são introduzidos como marcos ao longo do caminho de libertação, e amizade, gratidão e serviço aos outros são apresentados como elementos fundamentais da tarefa essencial de descobrir e dar corpo a nossa bondade e humanidade~inatas.
\stopbuffer

\startbuffer[backcover-institution]

Templo Zu Lai\crlf
e\crlf
BLIA — Associação Internacional Luz de Buda\crlf
Estrada Municipal Fernando Nobre, 1461\crlf
06705-490 – Cotia – SP\crlf
{\small\symbol[phone]} (11) 4612 2895\crlf
{\small\symbol[envelope]} zulai@templozulai.org.br\crlf
{\tfa\symbol[web]} www.templozulai.org.br
\stopbuffer

\startbuffer[front-flap]
O panorama da vida humana no novo milênio alterou-se radicalmente, e os novos tempos, com seu ritmo acelerado, parecem sufocar a humanidade nos mesmos problemas de todas as eras: a depressão, a angústia e o medo são o pão cotidiano de~muitos.

Como escapar desses sofrimentos e alcançar a alegria e a felicidade?
Este livro oferece soluções práticas ao alcance de todos para esses problemas.
Ao longo de seus capítulos, de maneira harmoniosa e sempre apelando à razão humana, o autor mostra como o Budismo oferece orientações na vida cotidiana para despertar do estado de sono letárgico em que nos encontramos, desvelando as possibilidade grandiosas encerradas na existência humana.
E por ter a vida humana como fator central, o Budismo, com sua ética e seu equilíbrio, sempre pode atender as necessidades dos tempos modernos e constituir um farol para os tempos futuros.

Assim, mesmo numa sociedade moderna em que as tecnologias modernas exigem de nós verdadeiros malabarismos para conseguirmos manter-nos, podemos viver com contentamento em qualquer circunstância, seguindo o fluxo da vida, com os pés no chão, mas a cabeça no céu.
\stopbuffer
